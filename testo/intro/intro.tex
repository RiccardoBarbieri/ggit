\chapter*{Introduzione}
\label{chap:intro}
\addcontentsline{toc}{chapter}{Introduzione}
Un VCS (Version Control System) è un tipo di sistema utile a gestire le versioni di un progetto.
Nell'ambito dello sviluppo di progetti software, la funzionalità principale di un sistema di 
questo tipo è la possibilità di tenere traccia delle modifiche apportate al codice sorgente, ai file
di documentazione e ad altri contenuti di un progetto.\\
Senza l'utilizzo di un VCS, gli sviluppatori di un progetto potrebbero semplicemente mantenere diverse 
copie dei contenuti, classificandole a seconda della versione e delle modifiche apportate da una 
versione all'altra; un approccio simile tuttavia diventa ingestibile per progetti di grandi dimensioni,
in quanto si verrebbero a creare molteplici copie di file uguali o molto simili tra di loro.

In questo lavoro di tesi verrà illustrato e discusso il processo di sviluppo di un rudimentale VCS, 
chiamato GGit (ispirato a Git), basato su un database a grafi e implementato come una applicazione CLI
in Python 3.10.

I database a grafi sono database che fanno uso di strutture a grafi per immagazzinare informazioni.
Questo tipo di struttura è composta da due elementi fondamentali, nodi (o vertici) e archi, applicando
all'ambito dei database, i nodi sono utilizzati per rappresentare le entità, mentre gli archi
rappresentano le relazioni tra le entità.

Una delle proprietà più vantaggiose dei database a grafi utilizzati per questo progetto, ovvero 
LPG o labeled-property graph, consiste nel fatto che il tempo di attraversamento di un grafo 
è costante\cite{traversaltime}.